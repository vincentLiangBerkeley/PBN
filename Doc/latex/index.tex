An application for Simulation/\+Inverse Decomposition of Probabilistic Boolean Networks.\hypertarget{index_Introduction}{}\subsection{Introduction}\label{index_Introduction}
This is an open-\/source application that can be used in the research/analysis of P\+B\+N systems.

For users, you can inverse decompose your own transition matrix data and get the Entropy, Decomposition and Time data, or you can simulate random boolean matrices and weights for transition matrices.

For developer users, you can test your own algorithms using this A\+P\+I. Only the inverse algorithm part has to be implemented, all other functionalities are provided by the application itself.\hypertarget{index_Compilation}{}\subsection{Compilation}\label{index_Compilation}
This application now comes with the source file with makefile and is currently a console application working on U\+N\+I\+X-\/like systems.

If you are on Mac\+O\+S or Linux System, just unzip the \char`\"{}\+Analyzer.\+zip\char`\"{} file, then use the terminal to go into \char`\"{}\+Release\char`\"{} folder, type \char`\"{}make\char`\"{} to compile the binary. Then type \char`\"{}./\+Analyzer\char`\"{} to run the application.

If you are on Windows system, you should first download and install \char`\"{}cygwin\char`\"{} with \char`\"{}gcc, make\char`\"{} packages following the website After that, put the unzipped file in your home directory where your \char`\"{}cygwin\char`\"{} executable is. Then repeat the steps for Mac\+O\+S system.\hypertarget{index_Mannual}{}\subsection{Mannual}\label{index_Mannual}
When running the Analyzer application, first a menu will pop up and user are aked to choose a funcional module.
\begin{DoxyEnumerate}
\item Simulate data \+: This module will simulate test data and save the data in directory \char`\"{}\+Input/\+Simulated\char`\"{}. It is advised to input descriptive suffix when asked for.
\item Test provided data \+: This module will let the user test their own provided data. First make sure that your matrix data is in \char`\"{}.\+txt\char`\"{} format with columns separated by \char`\"{}\+Tabs\char`\"{} and rows separated by \char`\"{}new\+Lines\char`\"{}. The application has a \char`\"{}default mode\char`\"{} where it assumes you are testing \char`\"{}\+Provided data\char`\"{} rather than \char`\"{}\+Simulated data\char`\"{} and the algorithm used will be \char`\"{}\+O\+P\+T\+I\+M\+A\+L\char`\"{}. User can configure the mode according to their own need.
\item and 4. are batch testing modules, on \char`\"{}\+Simulated data\char`\"{} and \char`\"{}\+Provided data\char`\"{} respectively. The test data are prepared and upon chosen, the module will directly start running.
\end{DoxyEnumerate}\hypertarget{index_Developer}{}\subsection{Developer}\label{index_Developer}
If you want to use the application for testing your own inverse decomposition algorithm, you should follow the steps\+: Step 1 \+: Go to \char`\"{}\+Release/config.\+txt\char`\"{} and add the name of your algorithm in a new line before \char`\"{}\+\_\+\+D\+E\+F\+A\+U\+L\+T\+\_\+\char`\"{}; Step 2 \+: Go to \char`\"{}\+Interactor.\+h\char`\"{} and add your algorithm name to the Enum \char`\"{}\+Random\+Types\char`\"{}; Step 3 \+: Here since the class responsible for doing inverse decomposition is the \char`\"{}\+Iterator\char`\"{} class, there are basically two ways to add a new algorithm. One is write a subclass of \char`\"{}\+Iterator\char`\"{} class and override the \char`\"{}\+Iterate\char`\"{} method to include your new algorithm. The second will be just modifying the original \char`\"{}\+Iterate\char`\"{} method. But for the second way you might need to check the implementation of \char`\"{}iterate\+Once\char`\"{} and \char`\"{}choose\+Entry\char`\"{} methods. Both method requires your implementation of the algorithm in as separate function.\hypertarget{index_Design}{}\subsection{Design}\label{index_Design}
The major design used in this application is the \char`\"{}\+Model-\/\+View-\/\+Controller\char`\"{} design, where the \char`\"{}\+Interactor\char`\"{} serves as the \char`\"{}\+Abstract of Controller\char`\"{}. The \char`\"{}\+View\char`\"{} part is abstracted in an interface called \char`\"{}\+Displayer\char`\"{}. When moving to other platforms or building a G\+U\+I, the most important part to rewrite is the \char`\"{}\+Menu\char`\"{} which is a subclass of \char`\"{}\+Interactor\char`\"{} that implements the \char`\"{}\+Displayer\char`\"{} interface. Hence completing the task of user interaction.

The algorithms are encapsulated in the \char`\"{}\+Model\char`\"{} part consists of \char`\"{}\+Simulator\char`\"{} and \char`\"{}\+Iterator\char`\"{}, where \char`\"{}\+Simulator\char`\"{} class is responsible for simulating transition matrix and \char`\"{}\+Iterator\char`\"{} class is responsible for doing inverse decomposition.

For further details, or usage of classes during development, please refer to the documentation of this application that follows. 